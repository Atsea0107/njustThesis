
\chapter{引言}
\label{chap:introduction}

考虑到大多数用户可能并无\LaTeX{}使用经验,本模板将\LaTeX{}的复杂性尽可能地进行了封装,开放出简单的接口,以便于使用者可以轻易地使用,同时,对使用\LaTeX{}撰写论文所遇到的一些主要难题,如插入图片、文献索引等,进行了详细的说明,并提供了相应的代码样本,理解了上述问题后,对于初学者而言,使用此模板撰写其学文论文将不存在实质性的困难,所以,如果您是初学者,请不要直接放弃,因为同样作为初学者的我,十分明白让\LaTeX{}变得简单易用的重要性,而这正是本模板所体现的。

该南京理工大学学位论文模板\texttt{njustThesis}基于中科院大学学位论文\texttt{njustThesis}模板(https://github.com/mohuangrui/njustThesis)发展而来,\texttt{njustThesis}文档类的基础架构为ctexbook文档类。当前\texttt{njustThesis}模板基本满足最新的南京理工大学学位论文撰写要求和封面设定(http://gs.njust.edu.cn/a/xwgl/xwsq/20130508/82.html)。模板提供了pdflatex 或xelatex (默认,推荐) 编译方式,完美地支持中文书签、中文渲染、中文粗体显示、拷贝pdf中的文本到其他文本编辑器等特性,此外,对模板的文档结构进行了精心设计,撰写了编译脚本提高模板的易用性和使用效率。

宏包的目的是简化学位论文的撰写,模板文档的默认设定是十分规范的,从而论文作者可以将精力集中到论文的内容上,而不需要在版面设置上花费精力。 同时,在编写模板的\LaTeX{}文档代码过程中,作者对各结构和命令进行了十分详细的注解,并提供了整洁一致的代码结构,对文档的仔细阅读可以为初学的您提供一个学习\LaTeX{}的窗口。除此之外,整个模板的架构十分注重通用性,事实上,本模板不仅是南京理工大学学位论文模板,同时,也是使用\LaTeX{}撰写中英文article或book的通用模板,并为使用者的个性化设定提供了接口和相应的代码。

\section{系统要求}
\label{sec:sysRequire}

\texttt{njustThesis}宏包可以在目前大多数的\TeX{}系统中使用,例如C\TeX{}、MiK\TeX{}、te\TeX{}。考虑到大多数用户将是Windows使用者,推荐安装最新的C\TeX{}套装(2.9.1及其以上版本),C\TeX{}套装中包含了本模板中出的各类宏包,用户无需额外的设置即可使用。

\texttt{njustThesis}宏包通过\texttt{ctexbook}宏包来获得中文支持。\texttt{ctexbook}宏包提供了一个统一的中文\LaTeX{}书籍文档框架,底层支持CCT和CJK两种中文\LaTeX{}系统。

此外,\texttt{njustThesis}宏包还使用了宏包mathtools、amsthm、amsfonts、amssymb、bm、natbib和hyperref。目前大多数的\TeX{}系统中都包含有这些宏包。

目前已经测试的系统和版本为:
%% ++++++++++++++++++++++++++++++++++++++
\begin{enumerate}
\item texlive-2014(MacTeX on Mac OSX);

\item texlive-2014 + winEdt8.0 + SumatraPDF(Windows OS);

\item macTex-2015.

\end{enumerate}
%% ++++++++++++++++++++++++++++++++++++++

{\color{blue}{其中,针对2014版本的texlive需要首先配置中文字库,详细方法可以自行百度;2015版本之后ctex已经在发行版内完成了中文字库的配置,可以直接进行编译。另外,使用前需要自行设置中文参考文献标准,参考bib目录下GBT7714-2005NLang压缩文件说明。}}

\section{下载与使用}
\label{sec:howtouse}
njustThesis 模版包的最新版本可以从 https://github.com/jiec827/njustThesis, 网站下载。既可以直接通过网站右侧命令下载包,也可以通过命令行下载(推荐,但是需要安装git)

\begin{center}
  {\color{blue}{git clone https://github.com/jiec827/njustThesis}}
\end{center}


njustThesis 宏包包含4个文件夹和主函数myThesis.tex,另外附有说明文档和示例模版。文件架构如下:

0. LICENSE: GPL证书,开源性质

1. README.md: 项目介绍信息,包含使用和其他基本信息

2. HowToUseIt.pdf: 该项目生成的一个样本PDF文件(最近一次推送版本)

3. njustThesis.tex: 主函数

4. sty: (directory) 排版格式信息文件夹

5. tex: (directory) 论文内容文件夹,包含封面、摘要、章节、附录等

6. img: (directory) 论文中使用到的插图文件夹,包含学校logo和章节图片

7. bib: (directory) 参考文献文件夹,使用bibTeX格式

使用基本流程见READ.md文件或中文摘要中的描述。基本思路是,分别将封面、章节和附件中的内容更新到对应的*.tex文件,然后第一次编译刷新tex系统,在更新对应的引用文件系统,最后一次编译获得最终排版稿件。

值得提醒的是,宏包中提供的范例文档《HowToUseIt.pdf》,对使用过程经常需要使用到的命令行进行的例句,尤其是插入图片时会遇到的单列、多列问题,表格的三线、多线,交叉引用等问题,都通过具体实例给出。对于具体用户而言,可以直接采用替换文中引用文件名的方式快速入门使用。

由于笔者个人的时间精力有限,可能会有一些尚未想到的很多方面,往广大师生朋友多多交流互动!

\section{问题反馈}
\label{sec:FandQ}

\subsection{目前尚存在的问题}
\label{sec:remainingProblem}
截止此次编译版本,\texttt{njustThesis}尚未满足的设计项(设计标准参见第~\ref{app:format}章)包含:

1. 术语表的页眉尚未设置完整,因此未在编译中添加术语表;

2. 一级标题(chapter)的行距问题存在疑问,段前行距略显大;

3. 中文参考文献在列表时,作者(4个以上)后面的“et al”需要在*.bbl文件中手动替换为“等”。


用户在使用中遇到问题或者需要增加某种功能,可以和作者联系:

{\it{程杰 (Jie Cheng) \quad jie.cheng@aliyun.com}}

或者直接参与GitHub$^{\circledR}$开源项目,进行互动性的完善开发:

{\it{https://github.com/jiec827/njustThesis}}。

欢迎大家反馈自己的使用情况,使我们可以不断改进宏包。笔者在这还要感谢前人的大量工作(https://code.google.com/p/latex-njust808/),项目中大量的格式细节部分代码均拷贝自该项目。
