
\chapter{引言}
\label{chap:introduction}

考虑到大多数用户可能并无\LaTeX{}使用经验,本模板将\LaTeX{}的复杂性尽可能地进行了封装,开放出简单的接口,以便于使用者可以轻易地使用,同时,对使用\LaTeX{}撰写论文所遇到的一些主要难题,如插入图片、文献索引等,进行了详细的说明,并提供了相应的代码样本,理解了上述问题后,对于初学者而言,使用此模板撰写其学文论文将不存在实质性的困难,所以,如果您是初学者,请不要直接放弃,因为同样作为初学者的我,十分明白让\LaTeX{}变得简单易用的重要性,而这正是本模板所体现的。

该南京理工大学学位论文模板\texttt{njustThesis}基于中科院大学学位论文\texttt{ucasthesis}模板(https://github.com/mohuangrui/ucasthesis)发展而来,\texttt{njustThesis}文档类的基础架构为ctexbook文档类。当前\texttt{njustThesis}模板基本满足最新的南京理工大学学位论文撰写要求和封面设定(http://gs.njust.edu.cn/a/xwgl/xwsq/20130508/82.html)。模板提供了pdflatex 或xelatex (默认,推荐) 编译方式,完美地支持中文书签、中文渲染、中文粗体显示、拷贝pdf中的文本到其他文本编辑器等特性,此外,对模板的文档结构进行了精心设计,撰写了编译脚本提高模板的易用性和使用效率。

宏包的目的是简化学位论文的撰写,模板文档的默认设定是十分规范的,从而论文作者可以将精力集中到论文的内容上,而不需要在版面设置上花费精力。 同时,在编写模板的\LaTeX{}文档代码过程中,作者对各结构和命令进行了十分详细的注解,并提供了整洁一致的代码结构,对文档的仔细阅读可以为初学的您提供一个学习\LaTeX{}的窗口。除此之外,整个模板的架构十分注重通用性,事实上,本模板不仅是南京理工大学学位论文模板,同时,也是使用\LaTeX{}撰写中英文article或book的通用模板,并为使用者的个性化设定提供了接口和相应的代码。

\section{系统要求}

\texttt{ucasthesis}宏包可以在目前大多数的\TeX{}系统中使用,例如C\TeX{}、MiK\TeX{}、te\TeX{}。考虑到大多数用户将是Windows使用者,推荐安装最新的C\TeX{}套装(2.9.1及其以上版本),C\TeX{}套装中包含了本模板中出的各类宏包,用户无需额外的设置即可使用。

\texttt{ucasthesis}宏包通过\texttt{ctexbook}宏包来获得中文支持。\texttt{ctexbook}宏包提供了一个统一的中文\LaTeX{}书籍文档框架,底层支持CCT和CJK两种中文\LaTeX{}系统。

此外,\texttt{ucasthesis}宏包还使用了宏包mathtools、amsthm、amsfonts、amssymb、bm、natbib和hyperref。目前大多数的\TeX{}系统中都包含有这些宏包。

目前已经测试的系统和版本为:

a. texlive-2014(MacTeX);

b. Mac OSX 

\section{问题反馈}

用户在使用中遇到问题或者需要增加某种功能,可以和作者联系:

{\it{程杰 (Jie Cheng) \quad chengjie0827@gmail.com}}

欢迎大家反馈自己的使用情况,使我们可以不断改进宏包,或者直接参与GitHub$^{\circledR}$开源项目(https://github.com/jiec827/njustThesis)。

笔者在这还要感谢前人的大量工作(https://code.google.com/p/latex-njust808/),项目中大量的格式细节部分代码均拷贝自该项目。
