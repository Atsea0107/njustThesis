
\chapter{文档架构}
\label{chap:textStructure}

本章的目的是介绍\LaTeX{}的文本控制流程,即如何实现文本在各章节中的分布,以及章节内的交叉引用问题,用户可以根据自身对\LaTeX{}的熟悉程度适当地略过阅读。在了解了本章的内容后,用户即可快速实现文本内容的粘贴和复制,实现一个在文本内容满足基本需求的文档。

\section{章节控制流程}
\label{sec:controlFlow}
\subsection{全文流程}
从整体而言,实现整个文档的架构在{\it{myThesis.tex}}中,通过{\it{input}}命令对需要的章节进行输入,基本格式如下表格~\ref{tab:controlFlow}


\begin{table}
 \centering
  \caption{章节控制流程}        % title of Table
  \label{tab:controlFlow}    % label of Table
   \begin{tabular}{l}
   \hline
   \textbackslash begin\{document\}                \\
   \%\% start front page No.                       \\
   \textbackslash frontmatter                      \\   
   \textbackslash input\{tex/cover\}                 \\
   \%\%\%\%\%\%\%\%\%\%\%\%\%\%\%\%\%\%\%\%\%\%\%\%\%\%\%\%\%\% \\
   \%\% start main page No.                        \\
   \textbackslash mainmatter                       \\   
   \textbackslash input\{tex/ch1-introduction\}\%    \\
   \textbackslash input\{tex/ch2-guide\}\%           \\
   \textbackslash input\{tex/ch3-texStructure\}\%    \\
   \textbackslash input\{tex/ch4-eqnFigAndTab\}\%    \\
   \%\%\%\%\%\%\%\%\%\%\%\%\%\%\%\%\%\%\%\%\%\%\%\%\%\%\%\%\%\% \\
   \textbackslash input\{tex/appendix\}\%            \\
   \%\%\%\%\%\%\%\%\%\%\%\%\%\%\%\%\%\%\%\%\%\%\%\%\%\%\%\%\%\% \\
   \%\% start back page No.                        \\
   \textbackslash backmatter                       \\ 
   \textbackslash input\{tex/publications\}          \\
   \textbackslash bibliography\{bib/myRefs\}         \\
   \textbackslash end\{document\}                    \\
   \hline              
  \end{tabular}
\end{table}

全文流程分为三部分:front、main和back,三部分采用不同页码系统和页眉页脚,需要在代码中分别表示出各部分开始位置。对于用户而言,重要的在于main,根据实际文章要求在此处添加需要的章节引用。

\subsection{章节设置}
章节的设置分别通过关键字完成,按照章节的级别依此如表~\ref{tab:setSection}所示,关于文档中具体章节的关键词设置可以参看原宏包中tex文件夹下的实例文件。


\begin{table}
 \centering
  \caption{章节设置关键字}     % title of Table
  \label{tab:setSection}    % label of Table
  \begin{tabular}{cl}
    \hline
    章节级别        & 关键字     \\
    \hline
    1 / 章        & \textbackslash chapter \\
    2 / 节        & \textbackslash section \\
    3 / 子节      & \textbackslash  subsection \\
    表格名称       & \textbackslash caption\{章节设置关键字\} \\
    引用标签       & \textbackslash label\{sec:labelName\} \\
    \hline
  \end{tabular}
\end{table}


\section{交叉引用}
\subsection{公式、图表和插图引用}
\label{sec:refofFigAndTab}
交叉引用的前提是需要在定义章节、公式和图表的时候都对其进行命名标签(即\textbackslash label\{sec:labelName\}命令),在实际使用过程中通过标签进行引用。根据引用的特点可以将应用分成表~\ref{tab:citeType}中所示三类。

\begin{table}
 \centering
  \caption{交叉引用类型}       % title of Table
  \label{tab:citeType}    % label of Table
  \begin{tabular}{cl}
    \hline
    引用类型     & 关键字     \\
    \hline
    标签设置        & \textbackslash label\{marker\}  \\
    引用代号        & \textbackslash ref\{marker\}    \\
    引用页码        & \textbackslash pageref\{marker\} \\
    引用文献        & \textbackslash cite\{regLabel\} \\
    \hline
  \end{tabular}
\end{table}

{\bf{实例1:}}这里是对表格《交叉引用类型》的引用——表~\ref{tab:citeType}位于第~\pageref{tab:citeType}页,其标签为\textbackslash label\{tab:citeType\}。

另外,在编译的过程中首次编译全文后需要对引用项进行索引编译,

\begin{center}
  {\color{blue}makeindex myThesis.nlo -s nomencl.ist -o myThesis.nls}
\end{center}

再进行第二次编译后才能更新全文中的交叉引用项。

\subsection{文献引用}
\label{sec:citeRefs}

{\bf{实例2:}}这里是对文献《{\it{State-Space Representation of Aerodynamic Characteristics of an Aircraft at High Angles of Attack}}》的引用——文献\cite{Goman:state_aerodynamics}。这里文献采用bibTex格式。

同样,在编译的过程中首次编译全文后需要对参考文献进行索引编译,

\begin{center}
  {\color{blue}bibtex myThesis.aux}
\end{center}

再进行第二次编译后才能更新全文中的文献引用项。
